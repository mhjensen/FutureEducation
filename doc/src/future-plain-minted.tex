%%
%% Automatically generated file from DocOnce source
%% (https://github.com/hplgit/doconce/)
%%
%%


%-------------------- begin preamble ----------------------

\documentclass[%
oneside,                 % oneside: electronic viewing, twoside: printing
final,                   % or draft (marks overfull hboxes, figures with paths)
10pt]{article}

\listfiles               % print all files needed to compile this document

\usepackage{relsize,makeidx,color,setspace,amsmath,amsfonts}
\usepackage[table]{xcolor}
\usepackage{bm,microtype}

\usepackage[pdftex]{graphicx}

\usepackage[T1]{fontenc}
%\usepackage[latin1]{inputenc}
\usepackage{ucs}
\usepackage[utf8x]{inputenc}

\usepackage{lmodern}         % Latin Modern fonts derived from Computer Modern

% Hyperlinks in PDF:
\definecolor{linkcolor}{rgb}{0,0,0.4}
\usepackage{hyperref}
\hypersetup{
    breaklinks=true,
    colorlinks=true,
    linkcolor=linkcolor,
    urlcolor=linkcolor,
    citecolor=black,
    filecolor=black,
    %filecolor=blue,
    pdfmenubar=true,
    pdftoolbar=true,
    bookmarksdepth=3   % Uncomment (and tweak) for PDF bookmarks with more levels than the TOC
    }
%\hyperbaseurl{}   % hyperlinks are relative to this root

\setcounter{tocdepth}{2}  % number chapter, section, subsection

\usepackage[framemethod=TikZ]{mdframed}

% --- begin definitions of admonition environments ---

% --- end of definitions of admonition environments ---

% prevent orhpans and widows
\clubpenalty = 10000
\widowpenalty = 10000

% --- end of standard preamble for documents ---


% insert custom LaTeX commands...

\raggedbottom
\makeindex

%-------------------- end preamble ----------------------

\begin{document}

% endif for #ifdef PREAMBLE


% ------------------- main content ----------------------



% ----------------- title -------------------------

\thispagestyle{empty}

\begin{center}
{\LARGE\bf
\begin{spacing}{1.25}
Education for the future
\end{spacing}
}
\end{center}

% ----------------- author(s) -------------------------

\begin{center}
{\bf Morten Hjorth-Jensen${}^{1, 2}$} \\ [0mm]
\end{center}


\begin{center}
{\bf Hans Petter Langtangen${}^{3}$} \\ [0mm]
\end{center}


\begin{center}
{\bf Anders Malthe-Sorenssen${}^{1}$} \\ [0mm]
\end{center}


\begin{center}
{\bf Knut Morken${}^{4}$} \\ [0mm]
\end{center}

\begin{center}
% List of all institutions:
\centerline{{\small ${}^1$Department of Physics, University of Oslo}}
\centerline{{\small ${}^2$Department of Physics and Astronomy, Michigan State University, USA}}
\centerline{{\small ${}^3$Department of Informatics, University of Oslo and Simula Research Laboratory}}
\centerline{{\small ${}^4$Department of Mathematics, University of Oslo}}
\end{center}
    
% ----------------- end author(s) -------------------------

\begin{center} % date
September 2 2015
\end{center}

\vspace{1cm}


% !split
\subsection*{Master program in Computational Physics, Mathematics and Life Science}

% --- begin paragraph admon ---
\paragraph{}

bla bla about content
% --- end paragraph admon ---



% !split
\subsection*{Strategic importance}

The program will educate the next generation of cross-disciplinary
science students with the knowledge, skills, and values needed to pose
and solve current and new scientific, technological and societal
challenges. The program will lay the foundation for cross-disciplinary
educational, research and innovation activities at the Faculty. The
program will contribute to building a common cross-disciplinary
approach to the key strategic initiatives at the Faculty: Energy,
Materials, Life Science, and Enabling Technologies.

A particular strength of physics students is their ability to pose and
solve problems that combine physical insights with mathematical tools
and now also computational skills. This provides a unique combination
of applied and theoretical knowledge and skills. These features are invaluable 
for the development of multi-disciplinary educational and research programs. 
In this program we build on and
refine this philosophy.  The main focus is not to educate computer
specialists, but to educate students with a solid understanding in basic science
as well as an integrated knowledge on how  to use 
essential methods from computational science. This requires an
education that covers both the specific disciplines like physics, biology,
geoscience, mathematics etc with a strong background in computational science.

% !split
\subsection*{Scientific and educational motivation}


% --- begin paragraph admon ---
\paragraph{Applications of simulation.}
Numerical simulations of various systems in science are central to our
basic understanding of nature and technlogy.
The increase in computational power,
improved algorithms for solving problems in science as well as access
to high-performance facilities, allow researchers nowadays to study
complicated systems across many length and energy scales. Applications
span from studying quantum physical systems in nanotechnology and the
characteristics of new materials or subamotic physics at its smallest
length scale, to simulating galaxies and the evolution of the universe.
In between, simulations are key to understanding
cancer treatment and how the brain works,
predicting climate changes and this week's weather,
simulating natural disasters, semi-conductor devices,
quantum computers, as well as assessing risk in the insurance and
financial industry. These are just a few topics
already well covered at the University of Oslo and that can be
topics for coming thesis projects as well as research directions.
% --- end paragraph admon ---




% --- begin paragraph admon ---
\paragraph{Job market.}
A large number of the candidates from the four involved departments
get jobs where numerical simulations are central and essential. The proposed
program will raise the educational quality in this area, because
our candidates need a broader understanding of the possibilities
and limitations of computation-based problem solving.
% --- end paragraph admon ---



% !split
\subsection*{Multiscale modeling is the big open research question}


% --- begin paragraph admon ---
\paragraph{}
Today's problems, unlike traditional
science and engineering, involve complex systems with many distinct
physical processes. The wide open research topic of this century, both
in industry and at universities, is how to effectively couple
processes across different length and energy scales. Progress will
rely on a multi-disciplinary approach and therefore a need for
a multi-disciplinary educational program.
% --- end paragraph admon ---




% --- begin paragraph admon ---
\paragraph{}
The proposed program will foster candidates with the right
multi-disciplinary background and comutational thinking for
understanding today's simulation technology and its challenges.
% --- end paragraph admon ---






% !split
\subsection*{Career prospects}


% --- begin paragraph admon ---
\paragraph{}
Candidates who are capable of modeling and understanding complicated
systems in natural science, are in short supply in society.  The
computational methods and approaches to scientific problems students learn
when working on their thesis projects are very similar to the methods
they will use in later stages of their careers.  To handle large
numerical projects demands structured thinking and good analytical
skills and a thorough understanding of the problems to be solved. This
knowledge makes the students unique on the labor market.


The program has also a strong international element which allows students to
gain important experience from international collaborations in
science, with the opportunity to spend parts of the time spent on 
thesis work at research institutions abroad.
% --- end paragraph admon ---





% ------------------- end of main content ---------------


\printindex

\end{document}

