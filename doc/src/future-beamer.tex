
% LaTeX Beamer file automatically generated from DocOnce
% https://github.com/hplgit/doconce

%-------------------- begin beamer-specific preamble ----------------------

\documentclass{beamer}

\usetheme{red_plain}
\usecolortheme{default}

% turn off the almost invisible, yet disturbing, navigation symbols:
\setbeamertemplate{navigation symbols}{}

% Examples on customization:
%\usecolortheme[named=RawSienna]{structure}
%\usetheme[height=7mm]{Rochester}
%\setbeamerfont{frametitle}{family=\rmfamily,shape=\itshape}
%\setbeamertemplate{items}[ball]
%\setbeamertemplate{blocks}[rounded][shadow=true]
%\useoutertheme{infolines}
%
%\usefonttheme{}
%\useinntertheme{}
%
%\setbeameroption{show notes}
%\setbeameroption{show notes on second screen=right}

% fine for B/W printing:
%\usecolortheme{seahorse}

\usepackage{pgf,pgfarrows,pgfnodes,pgfautomata,pgfheaps,pgfshade}
\usepackage{graphicx}
\usepackage{epsfig}
\usepackage{relsize}

\usepackage{fancybox}  % make sure fancybox is loaded before fancyvrb

\usepackage{fancyvrb}
%\usepackage{minted} % requires pygments and latex -shell-escape filename
%\usepackage{anslistings}
%\usepackage{listingsutf8}

\usepackage{amsmath,amssymb,bm}
%\usepackage[latin1]{inputenc}
\usepackage[T1]{fontenc}
\usepackage[utf8]{inputenc}
\usepackage{colortbl}
\usepackage[english]{babel}
\usepackage{tikz}
\usepackage{framed}
% Use some nice templates
\beamertemplatetransparentcovereddynamic

% --- begin table of contents based on sections ---
% Delete this, if you do not want the table of contents to pop up at
% the beginning of each section:
% (Only section headings can enter the table of contents in Beamer
% slides generated from DocOnce source, while subsections are used
% for the title in ordinary slides.)
\AtBeginSection[]
{
  \begin{frame}<beamer>[plain]
  \frametitle{}
  %\frametitle{Outline}
  \tableofcontents[currentsection]
  \end{frame}
}
% --- end table of contents based on sections ---

% If you wish to uncover everything in a step-wise fashion, uncomment
% the following command:

%\beamerdefaultoverlayspecification{<+->}

\newcommand{\shortinlinecomment}[3]{\note{\textbf{#1}: #2}}
\newcommand{\longinlinecomment}[3]{\shortinlinecomment{#1}{#2}{#3}}

\definecolor{linkcolor}{rgb}{0,0,0.4}
\hypersetup{
    colorlinks=true,
    linkcolor=linkcolor,
    urlcolor=linkcolor,
    pdfmenubar=true,
    pdftoolbar=true,
    bookmarksdepth=3
    }
\setlength{\parskip}{0pt}  % {1em}

\newenvironment{doconceexercise}{}{}
\newcounter{doconceexercisecounter}
\newenvironment{doconce:movie}{}{}
\newcounter{doconce:movie:counter}

\newcommand{\subex}[1]{\noindent\textbf{#1}}  % for subexercises: a), b), etc

%-------------------- end beamer-specific preamble ----------------------

% Add user's preamble




% insert custom LaTeX commands...

\raggedbottom
\makeindex

%-------------------- end preamble ----------------------

\begin{document}

% endif for #ifdef PREAMBLE



% ------------------- main content ----------------------



% ----------------- title -------------------------

\title{Education for the future}

% ----------------- author(s) -------------------------

\author{Morten Hjorth-Jensen\inst{1,2}
\and
Anders Malthe-Sørenssen\inst{1}}
\institute{Department of Physics, University of Oslo\inst{1}
\and
Department of Physics and Astronomy, Michigan State University, USA\inst{2}}
% ----------------- end author(s) -------------------------

\date{September 2 2015,
% <optional titlepage figure>
}

\begin{frame}[plain,fragile]
\titlepage
\end{frame}

\begin{frame}[plain,fragile]
\frametitle{This talk is about how we perceive the role of education; present and future}

\begin{block}{}

\begin{itemize}
\item \textbf{Research-based education}, from undergraduate studies to a PhD: \href{{http://www.mn.uio.no/fysikk/english/research/groups/computational/index.html}}{The Computational Physics group at the University of Oslo} as example

\item Future challenges and directions
\end{itemize}

\noindent
\end{block}
\end{frame}

\begin{frame}[plain,fragile]
\frametitle{The role of computations, from education to society}

\pause
\begin{block}{}
\textbf{Computations of almost all systems in science are central to our
basic understanding of nature and technological advances}.
\end{block}

\begin{block}{Examples }
\begin{itemize}
\item quantum physical systems in nanotechnology and the characteristics of new materials

\item subamotic physics at its smallest length scale

\item simulating galaxies and the evolution of the universe

\item cancer treatment and how the brain works

\item predicting climate changes and this week's weather

\item simulating natural disasters

\item semi-conductor devices, quantum computers,

\item assessing risk in the insurance and financial industry

\item and many many more
\end{itemize}

\noindent
\end{block}
\end{frame}

\begin{frame}[plain,fragile]
\frametitle{Modeling and computations as a way to enhance algorithminc thinking}

\pause
\begin{block}{}
\textbf{Algorithm} :
A set of instructions to solve a problem.
% A finite set of unambiguous instructions that, given some set of initial condit#ions, can be performed in a prescribed sequence to achieve a certain goal.
\end{block}

\begin{block}{Algorithmic thinking: }

\begin{itemize}
\item Enhances instruction-based teaching

\item Introduces research-based teaching  from day one

\item Triggers further insights in math and other disciplines

\item Emphasizes validation and verification of scientific results, and integrates ethics

\item Ensures good working practices from day one!
\end{itemize}

\noindent
\end{block}
\end{frame}

\begin{frame}[plain,fragile]
\frametitle{What does computing mean?}

\begin{block}{}

\textbf{Computing means solving scientific problems using computers. It covers numerical as well as symbolic computing. Computing is also about developing an understanding of the scientific process by enhancing the algorithmic thinking when solving problems.}
\end{block}


\begin{block}{Computing competence is about: }

\begin{itemize}
\item derivation, verification, and implementation of algorithms

\item understanding what can go wrong with algorithms

\item overview of important, known algorithms

\item understanding how algorithms are used to solve complicated problems

\item reproducible science and ethics

\item algorithmic thinking for gaining deeper insights about scientific problems
\end{itemize}

\noindent
All these elements (and many more) aid students in maturing and gaining a better understanding of the scientific process.
\end{block}
\end{frame}

\begin{frame}[plain,fragile]
\frametitle{Computing and research-based education}

\pause
\begin{block}{}
A computational approach allows us to introduce research concepts and engage students in research from \emph{day one}.
\end{block}
\begin{block}{How do we define it? }
It is coupled to a direct participation in actual research and builds upon established
knowledge and insights about scientific methods.

\shortinlinecomment{hpl 1}{ Think this is unclear...better to just phrase it orally? }{ Think this is unclear...better }
\end{block}
\end{frame}

\begin{frame}[plain,fragile]
\frametitle{Research-based education}

\begin{block}{What should the education contain? }

\begin{itemize}
\item Theory + experiment + simulation is the norm in research and industry

\item Modeling of real, complex systems with no simple answers

\item Insight and understanding of fundamental principles and laws

\item Visualization, presentation, discussion, interpretation, and critical analysis of results

\item Development of a sound ethical attitude to own and other's work

\item Enhanced reasoning about the scientific method
\end{itemize}

\noindent
This is what we do in the \href{{http://www.mn.uio.no/fysikk/english/research/groups/computational/index.html}}{Computational Physics group at UiO}!
\end{block}
\end{frame}

\begin{frame}[plain,fragile]
\frametitle{\href{{http://www.mn.uio.no/fysikk/english/research/groups/computational/index.html}}{Computational Physics group at UiO}; our visions}

\begin{columns}
\column{0.6\textwidth}
\begin{block}{}
A particular strength of physics students is their ability to \textbf{pose and
solve problems} that combine \textbf{physical insights} with \textbf{mathematical tools}
and now also \textbf{computational skills}. This provides a unique combination
of applied and theoretical knowledge and skills. These features are invaluable
for the development of multi-disciplinary educational and research programs.
\end{block}

\column{0.4\textwidth}
% inline figure
\centerline{\includegraphics[width=1.0\linewidth]{fig-future/computer_nerd2.jpg}}



\end{columns}
\end{frame}

\begin{frame}[plain,fragile]
\frametitle{We develop a social and scientific learning environment}

\begin{block}{}
The main aim is that students should realize their own potentials and creative power

\begin{itemize}
 \item Students come with different dreams, ambitions, aspirations and topics they wish to study, our approach is to tailor the education to all these aspects

 \item Our motto: foster students who are better than their supervisors - that's progress!

 \item Students and teachers help each other

 \item Students with different backgrounds and needs can thrive socially and scientifically

 \item No competing environment, but a drive and enthusiam for sharing and developing knowledge
\end{itemize}

\noindent
\end{block}
\end{frame}

\begin{frame}[plain,fragile]
\frametitle{We develop a social and scientific learning environment}

\begin{block}{}
\begin{itemize}
\item We target bachelor, MSc and PhD students

\item Project-oriented work where students develop and mature their own ideas, with an individually tailored approach to each student

\item Office space with desktops to every student and large common room for recreational activities (meals, gaming, movies)

\item Many students collaborate on similar  thesis topics and \href{{http://www.dn.no/talent/2014/06/12/Utdannelse/sommervikar-ble-toppforsker}}{publish in top scientific journals}
\end{itemize}

\noindent
\end{block}
\end{frame}

\begin{frame}[plain,fragile]
\frametitle{Features of the Computational Physics group}

\begin{block}{}
\begin{itemize}
\item Our students have made significant contributions to  the \href{{http://www.mn.uio.no/english/about/collaboration/cse/}}{Computing in Science Education}  (UiO education prize in 2011) by developing exercises and participating in educational projects at the MN faculty

\item Our students have also developed educational \href{{http://www.mn.uio.no/fysikk/om/aktuelt/aktuelle-saker/2015/realfagsapper.html}}{tools and applications for understanding complicated physical problems}

\item A group of PhD students is now developing \href{{https://github.com/CINPLA/ibvcse}}{new textbooks for Computational Life Science}

\item 2005-2015: $> 60$ students have finalized their master's theses and 60\% have continued with PhD studies

\item Many students don't want to leave the group after finishing their studies
\end{itemize}

\noindent
\end{block}
\end{frame}

\begin{frame}[plain,fragile]
\frametitle{Investing in equipment for students}

\begin{block}{Using research funds for visualization tools }


% inline figure
\centerline{\includegraphics[width=0.7\linewidth]{fig-future/visualize.jpg}}



\end{block}
\end{frame}

\begin{frame}[plain,fragile]
\frametitle{Building a supercomputing cluster}

\begin{block}{We got (for free) the old supercomputer at UiO (TITAN) }


% inline figure
\centerline{\includegraphics[width=0.7\linewidth]{fig-future/uniforum-0.png}}



\end{block}
\end{frame}

\begin{frame}[plain,fragile]
\frametitle{Undergraduate student publishes in PNAS}

\begin{block}{Using research funds for visualization tools }


% inline figure
\centerline{\includegraphics[width=0.7\linewidth]{fig-future/pnas.png}}



\end{block}
\end{frame}

\begin{frame}[plain,fragile]
\frametitle{Multiscale modeling is the big open research question in the 21st century}

\begin{block}{}
\begin{itemize}
\item Present and future problems, unlike traditional
  science and engineering, involve complex systems with many distinct
  physical processes

\item The wide open research topic of this century, both in industry and at universities, is how to effectively couple processes across different length and energy scales

\item Progress will rely on a \emph{multi-disciplinary} approach
\end{itemize}

\noindent
\end{block}

\begin{block}{}
We need to foster candidates with the right
multi-disciplinary background and computational thinking!
\end{block}
\end{frame}

\begin{frame}[plain,fragile]
\frametitle{Examples of large scale simulations}

\begin{block}{Fluid dynamical simulations central in air industry.  Typical university courses which are taught address the physics of the lower left corner. }


% inline figure
\centerline{\includegraphics[width=0.6\linewidth]{fig-future/fig10.jpg}}


\end{block}
\end{frame}

\begin{frame}[plain,fragile]
\frametitle{Testing plane wings via massive numerical simulations}

\begin{block}{}
Fluid dynamical simulations central in air industry, wings tested.


% inline figure
\centerline{\includegraphics[width=1.0\linewidth]{fig-future/fig8.jpg}}


\end{block}
\end{frame}

\begin{frame}[plain,fragile]
\frametitle{The challenges for the future}

\begin{block}{}
We need to educate the next generation of
science students with the knowledge, skills, and values needed to pose
and solve current and new scientific, technological and societal
challenges.
\end{block}
\begin{block}{}
This will lay the foundation for cross-disciplinary
educational, research and innovation activities. It will contribute to building a common cross-disciplinary
approach to key strategic initiatives, with examples like \emph{Energy, Materials, Life Science, and Enabling Technologies}.
\end{block}

\shortinlinecomment{hpl 2}{ Repetitions on this slide, lengthy text - better to just phrase this in the delivery }{ Repetitions on this slide, }
\end{frame}

\begin{frame}[plain,fragile]
\frametitle{A new type of students}

\begin{block}{}
\textbf{Candidates who are capable of modeling and understanding complicated
systems, are in short supply in society}.
\end{block}
\begin{block}{}
The computational methods and approaches to scientific problems students learn
when working on their thesis projects are very similar to the methods
they will use in later stages of their careers.
\begin{itemize}
\item To handle large numerical projects demands structured thinking and good analytical skills and a thorough understanding of the problems to be solved.

\item This knowledge makes the students unique on the labor market, a labor market which in the years to come will experience heavy automatization and massive loss of jobs.
\end{itemize}

\noindent
\end{block}

\begin{block}{}
Computations (mastering and developing)  will play a central role in almost all aspects of scientific investigations and technological innovation
\end{block}

\shortinlinecomment{hpl 3}{ True, but at this point time is needed for the new institute... Suggest to remove this slide. }{ True, but at this }
\end{frame}

\begin{frame}[plain,fragile]
\frametitle{Create the Department for Computational Science!}

\begin{block}{}
UiO's strength in computational science (education and research)
has the potential to make UiO a top European university
\end{block}

\begin{block}{How to achieve it }
\begin{itemize}
\item Establish  a new center/department with focus on computational science and its applications to a wide range of fields (natural science, medicine, social sciences, humanities, applied research etc)

\item Hire ten young professors (age $< 40$) dedicated to innovative \emph{computational} research and education

\item Establish another ten professorships with  shared positions between the  new department and the discipline-specificdepartment (physics, chemistry, ...)
\end{itemize}

\noindent
\end{block}

\textbf{The process must start now} in order not to lose momentum.
\end{frame}

\begin{frame}[plain,fragile]
\frametitle{Our success builds on the Computing in Science Education project (UiO educational prize in 2011)}

\begin{block}{}
The results, insights, ideas and thoughts presented here, would have been impossible without the continuous interaction with colleagues in the \href{{http://www.mn.uio.no/english/about/collaboration/cse/}}{Computing in Science Education} project.

\begin{itemize}
\item Hans Petter Langtangen, Informatics and Simula Research Laboratory

\item Knut Mørken, Mathematics

\item Arnt Inge Vistnes, Physics

\item Oyvind Ryan, Mathematics

\item Solveig Kristensen and Annik Myhre, Deans of Education, MN faculty

\item Hanne Sølna, Director of studies MN faculty
\end{itemize}

\noindent
\end{block}
\end{frame}

\end{document}
